% !TEX program = xelatex

\documentclass[a4paper, twocolumn]{article}

% layout
\usepackage{ctex}
\usepackage{flushend}
\usepackage{abstract}
\usepackage{fancyhdr}
\usepackage[scale=0.8]{geometry}
% content
\usepackage{authblk}
\usepackage{graphicx}
\usepackage[backref]{hyperref}

\pagestyle{fancy}

\title{对种群动态的数学手段模式化分析方法建立暨对实际农作物生产场景预测分析}
\author{马宜远}
\author{孙嘉俊}
\author{王明哲 \thanks{通讯作者}}
\renewcommand \Authands{,}
\affil{大连第二十四中学}
\date{}

\begin{document}

\twocolumn[
    \maketitle
    \begin{abstract}
        生态学在农业生产中常被用于预测农作物产量。
        本文使用计算机分析方法,
        就不同生态因素对叶菜类农作物产量影响进行建模,
        得到了综合种内、种间竞争、天敌影响、土壤肥力等生态因素综合作用下的叶菜类农作物种群数量模型,
        最终得以精准预测、精准干预。
    \end{abstract}
]
\saythanks

农业是我国的基础产业,在我国有着悠久的历史。
新中国成立后,我国人口迅速增长,
但受客观因素影响,我国存在人均占有耕地面积少、耕地品质不足等问题,
农业产能难以满足人口增长的需要。
农业生产中,作物的产量往往是人们亟需关注的问题。
在高中生物选修二《种群的数量特征》中,
我们初步学习了“J形增长曲线”与“S形增长曲线”,
即指数增长曲线与逻辑斯蒂曲线。
为进一步深入、系统地学习,并应用到实际生活中,
我们开展了本次研究,结合生态学知识与信息技术手段,
建立有效的标准化预测模型,将借助传统人力经验难以量化计算的生态学因素,
使用数学语言精确表达,建立动态数学模型,并使用计算机进行数值分析。
考虑到农业的经济效益影响,我们选择叶菜类农作物进行建模,
并结合现实环境因素进行模拟。

% TODO: 补充内容,修改措辞,完成最终模型
\section{S 形曲线及其拓展模型}

\subsection{Logistic 增长模型}

\subsubsection{简述}
皮埃尔·费尔哈斯(P.F.Verhulst, 1804-1809)于 1838 年引入逻辑斯蒂方程。
其为描述资源有限的环境下种群增长的最简单的方程,
也是很多生态学模型的基础,即我们熟知的 S 形增长曲线。
\cite{primer_eco}

\subsubsection{数学模型描述}
由于资源有限和种内竞争的存在,我们不难理解逻辑斯蒂方程的微分形式:
$$ \frac{\mathrm{d} N}{\mathrm{d} t} = r N (1 - \frac{N}{K}) $$
此为一阶常微分方程。利用微积分知识,可得到如下通解:
$$ N_t = \frac{K}{1 + (\frac{K}{N_0} - 1) e^{-r t}} $$
\begin{figure}[h]
    \centering
    \includegraphics[width=\columnwidth]{./figures/logistic.pdf}
    \caption{Logistic 增长模型}
\end{figure}

逻辑斯蒂方程的简洁是毋庸置疑的。
然而,逻辑斯蒂方程在保证简洁性的同时也失去了其面对和处理复杂现实情况的能力。
在现实情况中,种间竞争凭着其对出生率、死亡率的直接影响,
毫无疑问成为了我们需要首要考虑的因素。
接下来,我们将在逻辑斯蒂的基础上,增加种间竞争的因素。

\subsection{Lotka-Volterra 竞争模型}

\subsubsection{简述}
二十世纪 20-30 年代,艾尔弗雷德·洛特卡(Alfred Lotka, 1880-1949)
独立提出了描述种间竞争的简单数学模型。
这些模型后来成为了生态学里面研究竞争的基础和框架。

\subsubsection{数学模型描述}
模型考虑的是两个竞争性物种的种群,分别设为 $N_1$, $N_2$。
每个种群均按照逻辑斯蒂方式增长,
有自己的内禀增长率 ($r_1$ 或 $r_2$) 和环境容纳量 ($K_1$ 或 $K_2$)。
与逻辑斯蒂模型一样, 种内竞争抑制种群增长:
$$ \frac{\mathrm{d} N_1}{\mathrm{d} t} = r_1 N_1 (\frac{K_1 - N_1 - \alpha N_2}{K_1}) $$
$$ \frac{\mathrm{d} N_2}{\mathrm{d} t} = r_2 N_2 (\frac{K_2 - N_2 - \beta N_1}{K_2}) $$
\begin{figure}[h]
    \centering
    \includegraphics[width=\columnwidth]{./figures/lotka-volterra.pdf}
    \caption{Lotka-Volterra 竞争模型}
\end{figure}

正如前文所述,一个种群对另一个种群的作用将会产生一种选择压力,
在这种选择压力下,会形成一系列的适应。
接下来我们研究植物——食草动物系统的动态模型。

\subsection{Lotka-Volterra 捕食模型}

\subsubsection{食草动物的引入}
现在我们把食草动物引入这一模型,只需在原方程上额外增加一项,
以表达受食草动物的啃食程度,因此方程可以写为:
$$ \frac{\mathrm{d} V}{\mathrm{d} t} = r_m (1 - \frac{V}{K}) - c_1 H (\frac{1 - e^{-d_1} V}{V}) $$

\subsubsection{食肉动物的引入}
如果把捕食动物猎杀食草动物引入上述模型,模型就会变得更加复杂。
其中的植物种群方程仍然保持不变,但描述食草动物增长的方程会因为加入
食肉动物而变得复杂起来:
$$ \frac{\mathrm{d} H}{\mathrm{d} t} = a_2 + c_2 (1 - e^{d_2} V) - f P (\frac{1 - e^{-d_3} H}{H}) $$
对于食肉动物,我们则采用一个新方程加以描述:
$$ \frac{\mathrm{d} P}{\mathrm{d} t} = -a_3 + c_3 (1 - e^{d_4} H) $$

这样一个模型虽然仍然简陋,
但已经可以对叶菜类农作物的产量做出初步的预测和较为准确的判断,
尤其表现出了产量变化的逐步过程,这对于农业生产也是极其重要的。

\section{计算机模拟对实际应用的结合}

\subsection{数学与农田的对话}
数学建模作为现实生活中的一个重要工具,
它用简洁的数学语言描述了事物的变化。
在理解种群动态的任务中,尤其是从理论模型到实际应用的场景,
我们仍然面对诸多挑战。

\subsection{计算机模拟结果对实际应用的作用}

\subsubsection{预见性}
原始的农业生产在有限的经验下,需要
克服不确定的气候、资源等因素,才能提升产
量。计算机的模拟可以使经验判断成为定量
计算。值得注意的是,这种预见性并不要求
模型的绝对准确。复杂计算机模型的校准需
要明确处理结构的误差——意识到模型只是
对现实的近似,结果本身包含一定误差,这
对决策者而言也是非常重要的信息。

\subsubsection{优化性}
如上文所述,模型只是对现实的近似。显
而易见,模型越逼近现实,其实用价值越高。
计算机模拟与优化算法的结合,为解决这类
问题提供了可行路径。优化算法可以自我改
进深度优化算法可以探索许多人类无法察觉
的微小因素从而做出更加精准的判断。

\subsubsection{普及性}
理想情况下,每一个农民都要掌握一定
的气象、经济学知识,能准确解读天气数据,
能优化资源配置,但现实不是这样的。计算
机模拟在降低专业知识应用门槛方面可以发
挥独特作用。农民不必理解模型内部的微分
方程,不必掌握编程技能就能获得准确可靠
的数据。

\section{结论与展望}
数学建模与计算机模拟的价值,终究要
由实践来检验。在这个意义上,模型永远只
是模型——是对现实的简化与近似。但这种
简化如果得当,就能帮助我们从复杂中识别
简单,从噪声中辨别信号,从不确定性中锚
定决策。这或许正是“对种群动态的数学手
段模式化分析”对农作物生产场景的最根本
贡献。当然了,我们也要看到研究的局限性。
从空间的角度看,如在 LV 竞争模型中,如
果我们拥有一个更加广阔的充足空间,就需
要考虑到种群的空间分布对于其数量变化的
影响。例如大型食肉动物对小型动物的威慑
作用。从时间的角度看,因其还不具备引入
长期复杂现实因素的能力,可能在更加广阔
的时间尺度上略显逊色。例如将时间线拉长,
我们可能要考虑到进化、演替等因素,以及
基因的变异、自然选择等等。短期内这些因
素不会对模拟结果造成太大影响,但是当时
间线拉长,它们对于预测结果的影响程度成
指数级增加,成为我们不能忽略不计的因素。
从社会的角度看,计算机模拟的能力仍然不
足。从人类影响的程度上看,就本研究的例
子而言,当我们进一步思考实际意义,我们
需要考虑到人工对害虫的扑杀等因素。人类
对自然的影响是丰富的,不仅仅体现在程度
上,更体现在形式上。每一个人的细微举动
都会对自然造成影响,当人流量增大,如在
考虑人口增长或是城市中公园的生态问题时,
这些影响也是不可忽略的因素。而计算机如
何体现各种各样形式的影响成为了我们面对
此类问题时的主要挑战。数学建模与计算机
模拟的价值,终究要由实践来检验。在这个
意义上,模型永远只是模型——是对现实的
简化与近似。但这种简化如果得当,就能帮
助我们从复杂中识别简单,从噪声中辨别信
号,从不确定性中锚定决策。这或许正是“对
种群动态的数学手段模式化分析”对农作物
生产场景的最根本贡献。我们相信在不远的
将来,计算机技术可以成熟的被运用到实际
生产生活当中,让人类的生活更加美好。

\bibliographystyle{plain}
\bibliography{reference}

\end{document}
